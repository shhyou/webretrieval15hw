\documentclass{article}
\usepackage{amsmath,amsfonts,amssymb}
\usepackage{enumerate}
\usepackage[margin=3cm]{geometry}
\begin{document}
\title{WebRetrieval Assignment 2, PageRank}
\date{}
\author{b00902107 Shu-Hung You}
\maketitle
\section{Implementation}
\paragraph{Data Structures} All data used in the implementation of
PageRank algorithm is in the following data type

\begin{verbatim}
data PageRank = PageRank { eps :: Double , damp :: Double
                         , numNodes :: Int
                         , invOutDegree :: UArray Int Double
                         , sinkNodes :: [Int]
                         , sinkNodesWeight :: Double
                         , inEdges :: Graph }
\end{verbatim}

A (sparse) graph $G$ is represented by an array of type
\texttt{g :: Array Int (UArray Int Int)}, where the $i$-th element \texttt{g!i}
stores the adjacent vertices of vertex $i$. The \texttt{inEdges} field stores
$G^T$ where $G$ is the given graph.

\texttt{sinkNodes} stores vertices with zero out-degree. This information is
utilized to avoid building out edges for sink nodes.

\paragraph{Algorithm} The algorithm is derived from the following formula.
Sink nodes are separated from non-sink nodes to avoid redundant computation.
Other parts are just the direct translation of the formula.

\[\begin{aligned}
p_u
&= 1-d + d\sum_{(v,u)\in E}\frac{p_v}{O_v} \\
&= 1-d + d\sum_{v\ne u,\deg(v)=0}\frac{p_v}{N-1} + d\sum_{(v,u)\in E}\frac{p_v}{\deg(v)} \\
&= \left(1-d + \frac{d}{N-1}\sum_{\deg(v)=0}\frac{p_v}{N-1}\right) + d\sum_{(v,u)\in E}\frac{p_v}{\deg(v)} - \frac{d}{N-1} p_u[\![ v=u ]\!]
\end{aligned}\]

\begin{verbatim}
nextRank :: PageRank -> Vector -> Vector
nextRank page rank = I.accumArray (+) baseValue (1, numNodes page) $
  (++)
    (map (\u -> (u, -sinkWeight*rank!u)) $ sinkNodes page) -- remove self-loops
    [ (u, sum [ rank!v * outWeight page!v | v <- I.elems (inEdges page!u) ])
    | u <- I.indices rank ]
  where !baseValue = 1 - damping + sinkWeight * (sum . map (rank!) . sinkNodes $ page)
        sinkWeight = sinkNodesWeight page
        damping = damp page
\end{verbatim}

\noindent The program simply iterates until the rank vector converges.

\begin{verbatim}
pageRank :: PageRank -> [Vector]
pageRank page = map snd . takeWhile ((> epsilon*epsilon) . l2norm2) $ zip ranks (tail ranks)
  where ranks = iterate (nextRank page) (I.listArray (1, numNodes page) [1.0..])
\end{verbatim}

\section{Discussion}
This implementation is evaluated against the three provided test cases on
\texttt{linux20}. The result is shown in the figure. Interestingly, all three
test cases take approximately 100 iterations to converge.
\begin{center}
  \begin{tabular}{|c|r|r|r|r|r|}
  \hline
  \textbf{test} & \textbf{nodes} & \textbf{non-sink nodes} & \textbf{iteration} & \textbf{time A} & \textbf{time B} \\
  \hline
  \texttt{cs.stanford.2004-10.graph} & 50184 & 41215 & 102 & 10.86s & 7.41s \\
  \hline
  \texttt{stanford-08-03.graph} & 350004 & 111549 & 107 & 82.27s & 60.80s \\
  \hline
  \texttt{03-2006-wk3.graph} & 1768330 & 1512273 & 121 & 1625.30s & 1542.63s \\
  \hline
  \end{tabular}
\end{center}
\paragraph{Evaluate $l^2$-norm less often.} Originally, the algorithm evaluates
$l^2$-norm in every iteration. The overall computation time is shown in the
\textbf{time A} column. By evaluate $l^2$-norm only every 5 iterations, the
computation time for small test cases dropped about 25\% as shown
in the \textbf{time B} column. However, since
 each iteration takes much longer time in the last test case, the time spent by
redundant iterations cancelled out the improvement.

\paragraph{A Possible Optimization.} Another possibility is to collapse the
sink nodes into only one node with weighted edges, and reconstruct the PageRank
for individual nodes after the iteration converges. This effectively replaces
the $\sum_{\deg(v)=0} p_v/N$ term in the formula by $p_{v^*}/N$ where $v^*$ is
the collpased node. I didn't implement this optimization since the bottleneck
of the Haskell implementation isn't in this part (and $85.5\%$ nodes in the
last test case are not sink nodes).
\end{document}
